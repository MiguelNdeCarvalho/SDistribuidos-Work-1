\documentclass[11pt]{article}

\usepackage[a4paper, total={16cm, 24cm}]{geometry}
\usepackage[portuguese]{babel}
\usepackage[utf8]{inputenc}
\usepackage{graphicx}
\usepackage{amsmath}
\usepackage{tikz}
    \usetikzlibrary{shadows}
\usepackage{booktabs}
\usepackage[colorlinks=true]{hyperref}
\usepackage{listings}
    \renewcommand\lstlistingname{Listagem}
    \lstset{numbers=left, numberstyle=\tiny, numbersep=5pt, basicstyle=\footnotesize\ttfamily, frame=tb,rulesepcolor=\color{gray}, breaklines=true}
\usepackage{blindtext}

% -------------------------------------------------------------------------------------------
\title
{
    \includegraphics[width=0.4\textwidth]{imgs/university.png}
    \\[0.1cm]
    \textbf{1º Trabalho Prático} \\
    Sistemas Distribuídos
}

\author
{
    \textbf{Professor:} José Saias \\
    \textbf{Realizado por:} Miguel de Carvalho (43108), João Pereira (42864) 
}
\date{\today}

% -------------------------------------------------------------------------------------------
%                                Body                                                       %
% -------------------------------------------------------------------------------------------

\begin{document}
\maketitle

% -------------------------------------------------------------------------------------------
\section{Introdução}

\hspace{0,5cm}Neste trabalho foi solicitado a realização de um programa que ajude a organizar o \textbf{Sistema
de Vacinação} de um país. \par

O \textbf{Sistema de Vacinação} consiste numa aplicação que permite inscrever utilizadores para a Vacinação,
marcar utilizadores como já vacinados entre outras funcionalidades que auxiliam a gestão da vacinação.

% -------------------------------------------------------------------------------------------
\section{Implementação}

\subsection{Interfaces Answer e Request}

\hspace{0,5cm} A classe \verb|Answer| e a classe \verb|Request| representam interfaces de comunicação 
usadas entre o cliente e o servidor. O cliente envia um pedido ao servidor (Request) que será 
interpretado pelo servidor e este irá responder (Answer) com um "Reply" que será de novo interpretado
no lado do cliente.

\subsection{Cliente}

A classe \verb|VaccineClient| é o cliente usado para comunicar com o servidor.
Serviços que são possíveis de requisitar ao servidor através do cliente:

\begin{itemize}
    \item Consulta de centros de vacinação;
    \item Consulta do comprimento da fila de espera num centro;
    \item Inscrição para vacinação num dos centros (indicando o nome, género e idade);
    \item Registar a realização de vacinação prevista para a inscrição com o ćodigo X (removendo o cidadão da filade espera do centro em que se encontrava), 
    ficando registada a data e tipo de vacina;
    \item Reportar existência de efeitos secundários para um cidadão antes vacinado com o código C;
    \item Listar nº total de vacinados por tipo de vacina e nº de casos com efeitos secundários por tipo de vacina. 
\end{itemize}

\subsection{VaccineImpl}
A classe \verb|VaccineImpl| contem a implementação dos serviços listados anteriormente. Esta implementação inclui
também o tratamento de dados e a consulta aos dados guardados na \textbf{BD}.

\subsection{VaccineServer} 

\hspace{0,5cm} A classe \verb|VaccineServer| é responsável por criar a conexão ao \textbf{RMI} e criar
a conexão à \textbf{BD} usando as credenciais da mesma que estão presentes num \textbf{ficheiro de propriedades}.


\subsection{PostgresConnector}

\hspace{0,5cm} A classe \verb|PostgresConnector| é responsável por realizar a conexão a uma \textbf{BD PostgreSQL} e
permite realizar operações na mesma.

\subsection{Vaccine}

\hspace{0,5cm} A interface \verb|Vaccine| implementa os metodos que o cliente pode usar para
aceder remotamente ao \textbf{servidor}.

% -------------------------------------------------------------------------------------------
\section{Execução}

\begin{itemize}
    \item 1º passo - importar o ficheiro \verb|db.sql| para a \textbf{base de dados};
    \item 2º passo - alterar as credenciais de acesso à \textbf{base de dados} no ficheiro
    \verb|credentials.properties| dentro da pasta \verb|resources|;
    \item 3º passo - compilar as classes todas com o comando \verb|make all|;
    \item 4º passo - executar o \textbf{RMIRegistry} através do comando \verb|make registry|;
    \item 5º passo - iniciar o servidor através do comando \verb|make server| num novo terminal;
    \item 6º passo - iniciar o cliente através do comando \verb|make client| num novo terminal.
\end{itemize}

% -------------------------------------------------------------------------------------------
\section{Conclusão}

\hspace{0,5cm}Assim com a realização deste trabalho conseguimos criar uma \textbf{Sistema}
(Servidor/Aplicação) que permite gerir um \textbf{Sistema de Vacinação}.

Em suma, com a realização deste trabalho ficámos muito mais esclarecidos de como funciona o
\textbf{RMI} em junção com a \textbf{serialização} e como funciona o conector do \textbf{PostgreSQL}
na linguagem \textbf{Java}.

Saliento também que neste trabalho aplicámos todo o conhecimento que adquirimos durante as aulas e que
ajudou a compreender alguns pontos que não tínhamos entendido tão bem.
% -------------------------------------------------------------------------------------------
\end{document}